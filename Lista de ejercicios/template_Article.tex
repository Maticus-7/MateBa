\documentclass[12pt, a4paper]{article}
\usepackage[spanish]{babel}
\usepackage[utf8]{inputenc}
\usepackage[T1]{fontenc}
\usepackage{amsmath, amssymb, amsthm}
\usepackage{graphicx}
\usepackage{geometry}
\usepackage{xcolor}
\usepackage{enumitem}
\usepackage{tcolorbox}
\usepackage{hyperref}

% Configuración de márgenes
\geometry{left=2.5cm, right=2.5cm, top=2.5cm, bottom=2.5cm}

% Configuración de colores
\definecolor{theoremcolor}{RGB}{230, 240, 255}
\definecolor{exercisecolor}{RGB}{255, 250, 230}
\definecolor{proofcolor}{RGB}{245, 255, 245}

% Definición de entornos personalizados

% Entorno para Teoremas
\tcolorboxenvironment{theorem}{
	enhanced,
	colback=theoremcolor,
	boxrule=0.5pt,
	colframe=blue!50!black,
	arc=3pt,
	breakable,
	before skip=12pt,
	after skip=12pt,
	left=8pt,
	right=8pt,
	top=4pt,
	bottom=4pt,
	fontupper=\normalfont
}

% Entorno para Proposiciones
\tcolorboxenvironment{proposition}{
	enhanced,
	colback=theoremcolor,
	boxrule=0.5pt,
	colframe=green!50!black,
	arc=3pt,
	breakable,
	before skip=12pt,
	after skip=12pt,
	left=8pt,
	right=8pt,
	top=4pt,
	bottom=4pt,
	fontupper=\normalfont
}

% Entorno para Lemas
\tcolorboxenvironment{lemma}{
	enhanced,
	colback=theoremcolor,
	boxrule=0.5pt,
	colframe=orange!50!black,
	arc=3pt,
	breakable,
	before skip=12pt,
	after skip=12pt,
	left=8pt,
	right=8pt,
	top=4pt,
	bottom=4pt,
	fontupper=\normalfont
}

% Entorno para Corolarios
\tcolorboxenvironment{corollary}{
	enhanced,
	colback=theoremcolor,
	boxrule=0.5pt,
	colframe=purple!50!black,
	arc=3pt,
	breakable,
	before skip=12pt,
	after skip=12pt,
	left=8pt,
	right=8pt,
	top=4pt,
	bottom=4pt,
	fontupper=\normalfont
}

% Entorno para Definiciones
\tcolorboxenvironment{definition}{
	enhanced,
	colback=yellow!5,
	boxrule=0.5pt,
	colframe=yellow!50!black,
	arc=3pt,
	breakable,
	before skip=12pt,
	after skip=12pt,
	left=8pt,
	right=8pt,
	top=4pt,
	bottom=4pt,
	fontupper=\normalfont
}

% Entorno para Ejercicios
\newenvironment{exercise}[1][]
{
	\begin{tcolorbox}[
		enhanced,
		colback=exercisecolor,
		boxrule=0.5pt,
		colframe=red!50!black,
		arc=3pt,
		breakable,
		before skip=12pt,
		after skip=12pt,
		left=8pt,
		right=8pt,
		top=4pt,
		bottom=4pt,
		title=\textbf{Ejercicio~\ifx&#1&%
			\theexercise%
			\else%
			#1%
			\fi},
		fonttitle=\bfseries,
		fontupper=\normalfont
		]
	}
	{
	\end{tcolorbox}
}

% Entorno para Ejemplos
\newenvironment{example}[1][]
{
	\begin{tcolorbox}[
		enhanced,
		colback=exercisecolor,
		boxrule=0.5pt,
		colframe=brown!50!black,
		arc=3pt,
		breakable,
		before skip=12pt,
		after skip=12pt,
		left=8pt,
		right=8pt,
		top=4pt,
		bottom=4pt,
		title=\textbf{Ejemplo~\ifx&#1&%
			\theexample%
			\else%
			#1%
			\fi},
		fonttitle=\bfseries,
		fontupper=\normalfont
		]
	}
	{
	\end{tcolorbox}
}

% Entorno para demostraciones (estilo caja)
\newenvironment{myproof}[1][Demostración]
{
	\begin{tcolorbox}[
		enhanced,
		colback=proofcolor,
		boxrule=0.5pt,
		colframe=gray!50!black,
		arc=3pt,
		breakable,
		before skip=12pt,
		after skip=12pt,
		left=8pt,
		right=8pt,
		top=4pt,
		bottom=4pt,
		title=\textbf{#1},
		fonttitle=\bfseries,
		fontupper=\normalfont
		]
		\renewcommand{\qedsymbol}{$\blacksquare$}
	}
	{
		\hfill$\blacksquare$
	\end{tcolorbox}
}

% Contadores
\newcounter{theorem}
\newcounter{proposition}
\newcounter{lemma}
\newcounter{corollary}
\newcounter{definition}
\newcounter{exercise}
\newcounter{example}

% Comandos para comenzar cada entorno
\theoremstyle{plain}
\newtheorem{theoreminner}{Teorema}[section]
\newtheorem{propositioninner}{Proposición}[section]
\newtheorem{lemmainner}{Lema}[section]
\newtheorem{corollaryinner}{Corolario}[section]

\theoremstyle{definition}
\newtheorem{definitioninner}{Definición}[section]

% Redefinir entornos para usar las cajas
\newenvironment{theorem}[1][]
{\refstepcounter{theorem}\begin{theoreminner}[#1]\hspace{-0.5em}}
	{\end{theoreminner}}

\newenvironment{proposition}[1][]
{\refstepcounter{proposition}\begin{propositioninner}[#1]\hspace{-0.5em}}
	{\end{propositioninner}}

\newenvironment{lemma}[1][]
{\refstepcounter{lemma}\begin{lemmainner}[#1]\hspace{-0.5em}}
	{\end{lemmainner}}

\newenvironment{corollary}[1][]
{\refstepcounter{corollary}\begin{corollaryinner}[#1]\hspace{-0.5em}}
	{\end{corollaryinner}}

\newenvironment{definition}[1][]
{\refstepcounter{definition}\begin{definitioninner}[#1]\hspace{-0.5em}}
	{\end{definitioninner}}

% Comando para resaltar texto importante
\newcommand{\highlight}[1]{\textbf{\textcolor{blue}{#1}}}

% Configuración de hipervínculos
\hypersetup{
	colorlinks=true,
	linkcolor=blue,
	filecolor=magenta,
	urlcolor=cyan,
	pdftitle={Material Didáctico de Matemáticas},
	pdfauthor={Tu Nombre}
}

% Título y autor
\title{\textbf{Material Didáctico de Matemáticas}}
\author{Profesor: Tu Nombre}
\date{\today}

\begin{document}
	
	\maketitle
	
	\tableofcontents
	
	\newpage
	
	\section{Introducción a la Lógica Matemática}
	
	\subsection{Conceptos Básicos}
	
	\begin{definition}
		Una \highlight{proposición} es una oración declarativa que es verdadera o falsa, pero no ambas.
	\end{definition}
	
	\begin{example}[Proposiciones]
		\begin{enumerate}[label=\alph*)]
			\item "2 + 2 = 4" es una proposición verdadera.
			\item "La Tierra es plana" es una proposición falsa.
			\item "¿Qué hora es?" no es una proposición.
		\end{enumerate}
	\end{example}
	
	\subsection{Teoremas Fundamentales}
	
	\begin{theorem}[Teorema de Pitágoras]
		En un triángulo rectángulo, el cuadrado de la hipotenusa es igual a la suma de los cuadrados de los catetos. Es decir, si $a$ y $b$ son las longitudes de los catetos y $c$ es la longitud de la hipotenusa, entonces:
		\[
		c^2 = a^2 + b^2
		\]
	\end{theorem}
	
	\begin{myproof}
		Consideremos un triángulo rectángulo con catetos $a$ y $b$, e hipotenusa $c$. Construyamos un cuadrado de lado $(a+b)$ y coloquemos cuatro triángulos idénticos dentro de él.
		
		El área del cuadrado grande es $(a+b)^2$. Esta área también puede calcularse como la suma del área del cuadrado pequeño (de lado $c$) más el área de los cuatro triángulos:
		
		\[
		(a+b)^2 = c^2 + 4\left(\frac{ab}{2}\right)
		\]
		
		Desarrollando:
		\[
		a^2 + 2ab + b^2 = c^2 + 2ab
		\]
		
		Simplificando:
		\[
		a^2 + b^2 = c^2
		\]
		
		Lo que completa la demostración.
	\end{myproof}
	
	\begin{corollary}
		En un triángulo rectángulo isósceles, donde $a = b$, la hipotenusa es $c = a\sqrt{2}$.
	\end{corollary}
	
	\begin{myproof}
		Si $a = b$, entonces por el Teorema de Pitágoras:
		\[
		c^2 = a^2 + a^2 = 2a^2
		\]
		Por lo tanto, $c = a\sqrt{2}$.
	\end{myproof}
	
	\subsection{Ejercicios Prácticos}
	
	\begin{exercise}[Aplicación del Teorema de Pitágoras]
		Un triángulo rectángulo tiene catetos de longitudes 3 cm y 4 cm. Calcula la longitud de la hipotenusa.
	\end{exercise}
	
	\textbf{Solución:} Aplicando el Teorema de Pitágoras:
	\[
	c^2 = 3^2 + 4^2 = 9 + 16 = 25 \Rightarrow c = 5 \text{ cm}
	\]
	
	\begin{exercise}[Problema inverso]
		Si la hipotenusa de un triángulo rectángulo mide 13 cm y uno de sus catetos mide 5 cm, ¿cuánto mide el otro cateto?
	\end{exercise}
	
	\begin{exercise}[Demostración]
		Demuestra que en cualquier triángulo rectángulo, la altura correspondiente a la hipotenusa divide al triángulo en dos triángulos semejantes al triángulo original.
	\end{exercise}
	
	\section{Álgebra Lineal}
	
	\subsection{Espacios Vectoriales}
	
	\begin{definition}[Espacio Vectorial]
		Un \highlight{espacio vectorial} sobre un campo $\mathbb{K}$ es un conjunto $V$ no vacío, dotado de dos operaciones:
		\begin{enumerate}[label=\roman*)]
			\item Suma vectorial: $+: V \times V \rightarrow V$
			\item Multiplicación por escalar: $\cdot: \mathbb{K} \times V \rightarrow V$
		\end{enumerate}
		que satisfacen los ocho axiomas de espacio vectorial.
	\end{definition}
	
	\begin{lemma}[Unicidad del vector nulo]
		En un espacio vectorial $V$, el vector nero es único.
	\end{lemma}
	
	\begin{myproof}
		Supongamos que existen dos vectores nulos $0_1$ y $0_2$ en $V$. Entonces:
		\begin{align*}
			0_1 &= 0_1 + 0_2 \quad \text{(porque } 0_2 \text{ es vector nulo)} \\
			&= 0_2 + 0_1 \quad \text{(por conmutatividad)} \\
			&= 0_2 \quad \text{(porque } 0_1 \text{ es vector nulo)}
		\end{align*}
		Por lo tanto, $0_1 = 0_2$.
	\end{myproof}
	
	\begin{proposition}[Propiedades de espacios vectoriales]
		En un espacio vectorial $V$ sobre $\mathbb{K}$, para todo $\alpha \in \mathbb{K}$ y todo $v \in V$:
		\begin{enumerate}[label=\alph*)]
			\item $0 \cdot v = 0$
			\item $\alpha \cdot 0 = 0$
			\item $(-1) \cdot v = -v$
		\end{enumerate}
	\end{proposition}
	
	\subsection{Ejercicios de Álgebra Lineal}
	
	\begin{exercise}[Comprobación de axiomas]
		Verifica si los siguientes conjuntos son espacios vectoriales sobre $\mathbb{R}$:
		\begin{enumerate}[label=\alph*)]
			\item $\mathbb{R}^2$ con las operaciones usuales
			\item El conjunto de matrices $2\times 2$ con entradas reales
			\item El conjunto de polinomios de grado menor o igual a 3
		\end{enumerate}
	\end{exercise}
	
	\begin{exercise}[Dependencia lineal]
		Determina si los siguientes vectores de $\mathbb{R}^3$ son linealmente independientes:
		\[
		v_1 = (1,2,3), \quad v_2 = (4,5,6), \quad v_3 = (7,8,9)
		\]
	\end{exercise}
	
	\section{Soluciones a Ejercicios Seleccionados}
	
	\subsection{Solución Ejercicio 2}
	Dado que $c = 13$ cm y $a = 5$ cm, aplicamos el Teorema de Pitágoras:
	\[
	b^2 = c^2 - a^2 = 169 - 25 = 144 \Rightarrow b = 12 \text{ cm}
	\]
	
	\subsection{Solución Ejercicio 4}
	Para $\mathbb{R}^2$:
	\begin{itemize}
		\item La suma de vectores es conmutativa y asociativa
		\item Existe vector nulo: $(0,0)$
		\item Todo vector tiene inverso aditivo
		\item La multiplicación por escalar distribuye
	\end{itemize}
	Por lo tanto, $\mathbb{R}^2$ sí es espacio vectorial sobre $\mathbb{R}$.
	
	\vspace{1cm}
	\begin{center}
		\textbf{¡Fin del material!}
	\end{center}
	
\end{document}\\
